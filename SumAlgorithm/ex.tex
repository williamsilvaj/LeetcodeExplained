\documentclass{article}

\title{Sum of All Numbers in an Array - Algorithm Explanation}
\author{Your Name}
\date{}

\begin{document}

\maketitle

\section*{Problem Description}

Given an array of integers, we want to calculate the sum of all numbers in the array.

\section*{Algorithm}

To solve this problem, we can use a simple iterative approach. Here is the algorithm:

\begin{enumerate}
  \item Initialize a variable \texttt{total} to 0.
  \item Iterate through each element \texttt{num} in the array.
  \item Add \texttt{num} to \texttt{total}.
  \item Return the value of \texttt{total}.
\end{enumerate}

\section*{Pseudocode}

Here is the pseudocode representation of the algorithm:

\begin{verbatim}
function sumOfNumbers(array):
    total = 0
    for num in array:
        total += num
    return total
\end{verbatim}

\section*{Example}

Let's consider an example array:

\[
\text{{array}} = [1, 2, 3, 4, 5]
\]

Using the algorithm, we calculate the sum as follows:

\[
1 + 2 + 3 + 4 + 5 = 15
\]

Therefore, for the given array, the sum of all numbers is 15.

\section*{Time Complexity}

The time complexity of this algorithm is \(O(n)\), where \(n\) is the number of elements in the array. This is because we need to iterate through each element once to calculate the sum.

\section*{Conclusion}

The algorithm provides a simple and efficient way to calculate the sum of all numbers in an array. By following the iterative approach, we can achieve linear time complexity. This algorithm can be useful in various applications where the sum of array elements is required.

\end{document}
